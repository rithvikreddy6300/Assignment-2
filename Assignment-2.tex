\documentclass[a4paper]{article}
\usepackage[utf8]{inputenc}

\title{Assignment-2}
\author{S. RITHVIK REDDY- cs20btech11049}
\date{}
\usepackage{amsmath}
\usepackage{amssymb}
\usepackage{amsfonts}
\usepackage{nopageno}
\usepackage[margin=1in]{geometry}
\usepackage{graphicx}
\usepackage{float}
\usepackage{multicol}
\usepackage{hyperref}
\setlength{\columnsep}{0.5cm}
\setlength{\parindent}{0em}
\usepackage{color}
\usepackage{comment}
\setlength{\columnseprule}{1pt}
\def\columnseprulecolor{\color{black}}



\begin{document}
\maketitle
\noindent
Download all python codes from here

\begin{multicols*}{2}
\noindent
\fbox{%
    \parbox{0.45\textwidth}{%
        \url{https://github.com/rithvikreddy6300/Assignment-2/blob/main/Assignment-2.py}
    }%
    }
    
\vspace{0.3cm}
and latex-tikz codes from  

\vspace{0.3cm}  
    
\fbox{%
    \parbox{0.45\textwidth}{%
        \url{https://github.com/rithvikreddy6300/Assignment-2/blob/main/Assignment-2.tex}
    }%
    }
   
\vspace{0.5cm}
\textbf{Gate Problem-77}
\vspace{0.5cm}

If a random variable X assumes only positive integral values, with the probability 
$$P(X=x)=\dfrac{2}{3}\left(\dfrac{1}{3}\right)^{x-1} , x=1,2,3,...$$

(A) $\dfrac{2}{9}$ \hspace{2 cm} (C) 1

\vspace{0.5 cm}
(B) $\dfrac{2}{3}$ \hspace{2 cm} (D) $\dfrac{3}{2}$

Then E(X) is ?

\vspace{0.5cm}
\textbf{SOLUTION}
\vspace{0.5cm}

Let X=\{0,1\} be a set of random variables of a Bernoulli's distribution with 0 representing a loss and 1 a win, probability of loosing =$\dfrac{1}{3}$ and probability of winning is $\dfrac{2}{3}$.

The given probability distribution P(x=i) can be seen as the probability of winning a game at $i^{th}$ try for the first time. Because to win at  $i^{th}$ try for the first time you have to loose for first i-1 times whose probability is $\left(\dfrac{1}{3}\right)^{i-1}$ and win at $i^{th}$ try whose probability is 2/3

Given that $P(x=i)=\dfrac{2}{3}\left(\dfrac{1}{3}\right)^{x-1} , x=1,2,3,...$
The expectation value of X represented by E(X) is given by
$$E(X)=\sum_{i=1}^{\infty} Pr(x=i)\times i$$

Let S=E(X),
\begin{align}
&\implies E(X)=S=\sum_{i=1}^{\infty} Pr(x=i)\times i\\
&\implies S=\sum_{i=1}^{\infty} \dfrac{2}{3}\left(\dfrac{1}{3}\right)^{i-1} \times i \label{eq_2}   \\
&\implies S=\dfrac{2}{3}+\sum_{i=2}^{\infty} \dfrac{2}{3}\left(\dfrac{1}{3}\right)^{i-1} \times i  \label{eq_3}
\end{align}
Multiplying (\ref{eq_2}) with  $\dfrac{1}{3}$ on both sides gives
\begin{align}
&\dfrac{1}{3}S=\sum_{i=1}^{\infty} \dfrac{2}{3}\left(\dfrac{1}{3}\right)^{i} \times i \label{eq_4}
\end{align}

In (\ref{eq_3})	$\sum_{i=1}^{\infty} \dfrac{2}{3}\left(\dfrac{1}{3}\right)^{i} \times i$ can be written as $\sum_{i=2}^{\infty} \dfrac{2}{3}\left(\dfrac{1}{3}\right)^{i-1} \times (i-1)$
\begin{align}
& \implies \dfrac{1}{3}S=\sum_{i=2}^{\infty} \dfrac{2}{3}\left(\dfrac{1}{3}\right)^{i-1} \times (i-1) \label{eq_5} \\
\text{(\ref{eq_3})-(\ref{eq_5}) gives :}& \dfrac{2}{3}S=\dfrac{2}{3}+\sum_{i=2}^{\infty} \dfrac{2}{3}\left(\dfrac{1}{3}\right)^{i-1} \times (i-(i-1))\\
&\implies  \dfrac{2}{3}S=\dfrac{2}{3}+\sum_{i=2}^{\infty} \dfrac{2}{3}\left(\dfrac{1}{3}\right)^{i-1}\\
& \implies S=1+\sum_{i=1}^{\infty}\left(\dfrac{1}{3}\right)^{i}\\
& \implies S=1+\dfrac{1/3}{1-\dfrac{1}{3}}\\
& \implies S=\dfrac{3}{2} \label{eq_10}
\end{align}
From (\ref{eq_10}) we can say that the expectation value of X given by E(X)=S=$\dfrac{3}{2}$		
\textbf{(Option D).}


\end{multicols*}


\end{document}